\documentclass[a4paper, 12pt]{article}

\usepackage{amsmath} %Todos los paquetes de matematicas
\usepackage{amsthm}
\usepackage{amssymb}
\usepackage{amsfonts}
\usepackage{amssymb}
\usepackage[utf8]{inputenc}
\usepackage[spanish, es-lcroman]{babel}
\usepackage{wrapfig} %Figuras flotantes
\usepackage{parselines}
\usepackage{enumitem}
\usepackage{xcolor}
\usepackage{graphicx}
\usepackage{subcaption}
\usepackage{pgfplots}
\usepackage{hyperref}
\usepackage{textcomp}
\usepackage[left=2cm,right=2cm,top=2cm,bottom=2cm]{geometry}


\providecommand{\abs}[1]{\lvert#1\rvert} %Valor absoluto
\providecommand{\norm}[1]{\lVert#1\rVert} %Norma

\newtheorem*{lema*}{Lema}
\newtheorem{teorema}{Teorema}
\newtheorem*{teorema*}{Teorema}
\newtheorem{corolario}{Corolario}
\renewcommand{\qedsymbol}{\(\blacksquare\)}

\providecommand{\abs}[1]{\lvert#1\rvert} %Valor absoluto
\providecommand{\norm}[1]{\lVert#1\rVert} %Norma


\title{\textbf{Objetivos de aprendizaje Tema 6} \\ \textit{Análisis Matemático II}}
\author{Javier Gómez López}
\date{\today}

\begin{document}

\maketitle

\begin{enumerate}[label=\textbf{\arabic*}.]

\item Conocer y comprender las siguientes definiciones:

\begin{enumerate}[label=\textit{\alph*)}]
	\item Integral de una función simple positiva
	
	Sea \(s\) una función simple positiva, y consideremos su descomposición canónica:
	\begin{equation}\label{uno}
	s = \sum_{k=1}^{p} \alpha_k \chi_{A_k} \qquad \text{donde} \qquad p \in \mathbb{N}, \quad \alpha_1, \dotsc, \alpha_p \in \mathbb{R}_0^+, \quad A_1, \dotsc, A_p \in \mathcal{M}
	\end{equation}
	
	Para cada conjunto medible \(E \subset \Omega\), se define la \textbf{integral} de \(s\) sobre \(E\) mediante la igualdad
	\[
		\int_{E} s = \sum_{k=1}^{p} \alpha_k \lambda (E \cap A_k) 
	\]
	
	Para extender esta primera definición, sólo necesitamos tres propiedades:
	\begin{itemize}
		\item \textbf{Homogeneidad.} \textit{Si s es una función simple positiva, se tiene:}
		\[
			\int_E \rho s = \rho \int_E s \qquad \forall \rho \in \mathbb{R}_0^+, \quad \forall E \in \mathcal{M}
		\]
		
		\item \textbf{Aditividad.} \textit{Si s es una función simple positiva, la función \(\varphi: \mathcal{M} \to [0,\infty ] \) definida por}
		\[
			\varphi (E) = \int_E s \qquad \forall E \in \mathcal{M}
		\]
		\textit{es \(\sigma\)-aditiva y verifica que \(\varphi (\emptyset) = 0\), luego es una medida.}
		
		\item \textbf{Crecimiento.} \textit{Si s,t son funciones simples positivas y \(E \in \mathcal{M}\), entonces:}
		\[
			s(x) \leq t(x) \quad \forall x \in E \Rightarrow \int_E s \leq \int_E t
		\]
	\end{itemize}
	
	\medskip
	
	\item Integral de una función medible positiva
	
	Volvemos a trabajar en un conjunto medible \(\Omega \subset \mathbb{R}^N\), para estudiar la integral de una función \(f \in \mathcal{L}^+ (\Omega)\), esto es, una función medible \(f: \Omega \to [0, \infty]\).
	
	Se define la \textbf{integral de una función medible positiva} \(f: \Omega \to [0, \infty]\), sobre un conjunto medible \(E \subset \Omega\), mediante la igualdad:
	\begin{equation}\label{tres}
		\int_E f = \sup \left\{ \int_E s: s \in \mathcal{S}^+, \quad s(x) \leq f(x) \quad \forall x \in E \right\}
	\end{equation}
	donde \(\mathcal{S}^+\) es el conjunto de las funciones simples positivas.
	
	De nuevo, para extender esta definición, necesitamos estas tres propiedades:
	\begin{itemize}
		\item \textbf{Crecimiento.} \textit{Si f y g son funciones medibles positivas y \(E \in \mathcal{M} \cap \mathcal{P}(\Omega)\), se tiene:}
		\[
			f(x) \leq g(x) \quad \forall x \in E \Rightarrow \int_E f \leq \int_E g
		\]
		
		\item \textbf{Homogeneidad.} \textit{Si f es una función medible positiva, tiene:}
		\[
			\int_E \rho f = \rho \int_E f \qquad \forall \rho \in  \mathbb{R}_0^+, \quad \forall E \in \mathcal{M} \cap \mathcal{P}(\Omega)
		\]
		
		\item \textbf{Localización.} \textit{Si f es una función medible positiva, se tiene:}
		\[
			\int_E f = \int_{\Omega} \chi_E f \qquad \forall E \in \mathcal{M} \cap \mathcal{P} (\Omega)
		\]
	\end{itemize}
\end{enumerate}

\item Conocer y comprender el enunciado de los siguientes resultados:

\begin{enumerate}[label=\textit{\alph*)}]
	\item Integración de la suma de una serie
	
	A partir del teorema de aproximación de Lebesgue, obtendremos las propiedas más relevantes de la integral. Empezamos por su aditividad respecto al integrando. Conviene aclarar que, si \(\{f_n\}\) es una sucesión de funciones de \(\Omega\) en \([0, \infty]\), siempre tiene sentido considerar la función \(f: \Omega \to [0, \infty]\) definida por \(f(x) = \sum_{n=1}^{\infty} f_n(x)\) para todo \(x \in \Omega\). Es natural decir que \(f\) es la \textit{suma de la serie} \(\sum_{n \geq 1}^{\infty} f_n\). Enunciamos el siguiente resultado:
	\begin{itemize}
		\item \textbf{Integral de la suma de una serie.} \textit{Para cualquier sucesión \(\{f_n\}\) de funciones medibles positivas, se tiene:}
		\[
			\int_{\Omega} \sum_{n=1}^{\infty} f_n = \sum_{n=1}^{\infty} \int_{\Omega} f_n
		\]
	\end{itemize}
	
	\medskip
	
	\item Aditividad de la integral como función del conjunto sobre el que se integra
	
	Junto con la propiedad anteriormente enunciada, podemos ya enunciar la aditividad de la integral de una función medible positiva, que hasta ahora sólo conocíamos para funciones simples positivas:
	\begin{itemize}
		\item \textbf{Aditividad.} \textit{Si f es una función medible poisitiva, definiendo}
		\[
			\varphi (E) = \int_E f \qquad \forall E \in \mathcal{M} \cap \mathcal{P} (\Omega)
		\]
		\textit{se obtiene una función \(\varphi: \mathcal{M} \cap \mathcal{P} (\Omega) \to [0, \infty]\), que es \(\sigma\)-aditiva y verifica que \(\varphi ( \emptyset) = 0\), luego es una medida en \(\Omega\).}
	\end{itemize}
\end{enumerate}

\bigskip

\item Conocer y comprender el teorema de la convergencia monótona, incluyendo su demostración.

\begin{teorema*}[Convergencia monótona]
Sea \(\{f_n\}\) una sucesión creciente de funciones medibles positivas, y sea \(f(x) = \lim_{n \to \infty} f_n (x)\) para todo \(x \in \Omega\). Se tiene entonces:
\[
	\int_{\Omega} f = \lim_{n \to \infty} \int_{\Omega} f_n
\]
\end{teorema*}

\begin{proof}
Como \(\{f_n\}\) es creciente, la correspondiente sucesión de integrales también lo es, luego converge. Existe por tanto el límite que aparece en el segundo miembro de la igualdad buscada, al que denotaremos por \(L \in [0, \infty]\). Además se tiene \(f_n \leq f\) para todo \(n \in \mathbb{N}\), luego el crecimiento de la integral también nos dice que 
\[
	\int_{\Omega} f_n \leq \int_{\Omega} f \quad \forall n \in \mathbb{N}, \qquad \text{de donde} \qquad L \leq \int_{\Omega} f
\]

Para la otra desigualdad basta probar que, si \(s \in \mathcal{S}^+\) verifica que \(s(x) \leq f(x)\) para todo \(x \in \Omega\), la integral de \(s\) sobre \(\Omega\) es menor o igual que \(L\). Fijado \(\rho \in \mathbb{R}\) con \(0 < \rho < 1\), para cada \(n \in \mathbb{N}\) definimos
\[
	E_n = \{ x \in \Omega : f_n(x) \geq \rho s(x)\}
\]
y comprobamos que \(E_n \in \mathcal{M}\). Para ello, escribimos la descomposición canónica de \(s\) igual que en (\ref{uno}). Como \(f_n\) es medible, se tiene que
\[
	A_k \cap E_n = A_k \cap \{x \in \Omega: f_n(x) \geq \rho \alpha_k\} \in \mathcal{M} \quad \forall k \in \Delta_p, \qquad \text{luego} \qquad E_n = \biguplus_{k=1}^{p} (A_k \cap E_n) \in \mathcal{M}
\]

Como \(\{f_n\}\) es creciente, tenemos también que \(E_n \subset E_{n+1}\) para todo \(n \in \mathbb{N}\) y de hecho vamos a ver que \(\{E_n\} \nearrow \Omega\). Si \(s (x) = 0\) se tiene obviamente \(x \in E_n\) para todo \(n \in \mathbb{N}\). En otro caso, como \(\{f_n (x) \} \nearrow f(x) \geq s(x) > \rho s(x)\), existe \(n \in \mathbb{N}\) con \(x \in E_n\).

Fijado \(n \in \mathbb{N}\), como \(\rho s(x) \leq f_n(x)\) para todo \(x \in E_n\), usando la homogeneidad de la integral de \(s\), dos veces el crecimiento de la integral, y la definición de \(L\), obtenemos
\[
	\rho \int_{E_n} s = \int_{E_n} \rho s \leq \int_{E_n} f_n = \int_{\Omega} \chi_{E_n} f_n \leq \int_{\Omega} f_n \leq L
\]

Usamos ahora que la integral de \(s\), como función del conjunto sobre el que se integra es una medida, luego es crecientemente continua, para obtener:
\[
\rho \int_{\Omega} s = \rho \lim_{n \to \infty} \int_{E_n} s \leq L
\]

Pero esta desigualdad es válida para todo \(\rho \in ]0,1[\), luego la integral de \(s\) sobre \(\Omega\) es menor igual que \(L\), como queríamos demostrar.
\end{proof}

\end{enumerate}

\end{document}
