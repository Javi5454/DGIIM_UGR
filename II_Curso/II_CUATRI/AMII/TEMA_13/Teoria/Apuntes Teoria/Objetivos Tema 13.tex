\documentclass[a4paper, 12pt]{article}

\usepackage{amsmath} %Todos los paquetes de matematicas
\usepackage{amsthm}
\usepackage{amssymb}
\usepackage{amsfonts}
\usepackage{amssymb}
\usepackage[utf8]{inputenc}
\usepackage[spanish, es-lcroman]{babel}
\usepackage{wrapfig} %Figuras flotantes
\usepackage{parselines}
\usepackage{enumitem}
\usepackage{xcolor}
\usepackage{graphicx}
\usepackage{subcaption}
\usepackage{pgfplots}
\usepackage{hyperref}
\usepackage{textcomp}
\usepackage[left=2cm,right=2cm,top=2cm,bottom=2cm]{geometry}


\providecommand{\abs}[1]{\lvert#1\rvert} %Valor absoluto
\providecommand{\norm}[1]{\lVert#1\rVert} %Norma

\newtheorem*{lema*}{Lema}
\newtheorem{teorema}{Teorema}
\newtheorem*{teorema*}{Teorema}
\newtheorem{corolario}{Corolario}
\renewcommand{\qedsymbol}{\(\blacksquare\)}

\providecommand{\abs}[1]{\lvert#1\rvert} %Valor absoluto
\providecommand{\norm}[1]{\lVert#1\rVert} %Norma


\title{\textbf{Objetivos de aprendizaje Tema 13} \\ \textit{Análisis Matemático II}}
\author{Javier Gómez López}
\date{\today}

\begin{document}

\maketitle

\begin{enumerate}[label=\textbf{\arabic*}.]
	
	\item Conocer y comprender el enunciado de los siguientes resultados, así como la forma en que se usan en la práctica:
	
	\begin{enumerate}[label=\textit{\alph*)}]
		\item Teorema de Tonelli, con especial atención al caso particular de una función característica.
		
		En lo que sigue, fijamos \(p,q \in  \mathbb{N}\), para tomar \(N = p + q\). Sea \(Z\) un conjunto no vacío arbitrario, aunque sólo usaremos \(Z = \mathbb{R}\) y \(Z = [0,\infty]\). Dada una función \(f: \mathbb{R}^N \to Z\), para cada \(x \in \mathbb{R}^p\) definimos una función \(f_x : \mathbb{R}^q \to Z\) escribiendo
		\[
			f_x (y) = f(x,y) \qquad \forall y \in \mathbb{R}^q
		\]
		
		Suele decirse que \(f_x\) es la \textbf{sección vertical} de \(f\) en el punto \(x\). Análogamente, fijado \(y \in \mathbb{R}^q\), definimos la \textbf{sección horizontal} de \(f\) en el punto \(y\), que es la función \(f^y : \mathbb{R}^p \to Z\) dada por
		\[
			f^y (x) = f(x,y) \qquad \forall x \in \mathbb{R}^p
		\]
		
		Enunciamos ahora el teorema buscado:
		\begin{teorema*}[Tonelli]
		Si \(f: \mathbb{R}^N \to [0,\infty]\) es una función medible positiva, se tiene:
		\begin{enumerate}[label=\textit{(\roman*)}]
			\item La función \(f_x\) es medible p.c.t \(x \in \mathbb{R}^p\), y anáogamente, \(f^Y\) es medible p.c.t \(y \in \mathbb{R}^q\)
			\item Las función \(\varphi\) y \(\Psi\), definidas c.p.d en \(\mathbb{R}^p\) y \(\mathbb{R}^q\) respectivamente, por
			\[
				\varphi (x) = \int_{\mathbb{R}^q} f(x,y) dy \quad \text{p.c.t} \quad x \in \mathbb{R}^p \qquad \text{y} \qquad \Psi (y) = \int{\mathbb{R}^p} f(x,y) dx \quad \text{p.c.t} \quad y \in \mathbb{R}^q
			\]
			son medibles y verifican que:
			\[
				\int_{\mathbb{R}^N} f(x,y) d(x,y) = \int_{\mathbb{R}^p} \varphi (x) dx = \int_{\mathbb{R}^q} \Psi (y) dy
			\]
		\end{enumerate}
		\end{teorema*}
		
		\medskip
		
		Estudiemos ahora el caso en el que trabajemos con una función característica. Dado un conjunto \(E \subset \mathbb{R}^N\), pensemos ahora lo que debe ocurrir para que se tenga que \(\chi_E \in \mathcal{F}\), donde \(\mathcal{F}\) es la familia de las funciones que cumples las dos condiciones del teorema de Tonelli. Fijado \(x \in \mathbb{R}^p\), la sección vertical \((\chi_E)_x\) es la función característica del conjunto dado por
		\[
			E_x = \{y \in \mathbb{R}^q : (x,y) \in E\} \subset \mathbb{R}^q
		\]
		Es natural decir que \(E_x\) es la \textbf{sección vertical} de \(E\) en el punto \(x\). Análogamente, dado \(y \in \mathbb{R}^q\), la \textbf{sección horizontal} de \(E\) en el punto \(y\) es el conjunto \(E^y = \{ x \in \mathbb{R}^p : (x,y) \in E\} \subset \mathbb{R}^p\). Reenunciemos ahora el teorema:
		
		\begin{teorema*}
		Para todo conjunto medible \(E \subset \mathbb{R}^N\) se tiene:
		\begin{enumerate}[label=(\textit{\roman*})]
			\item La sección vertical \(E_x \subset \mathbb{R}^q\) es medible p.c.t \(x \in \mathbb{R}^p\) y la sección horizontal \(E^y \subset \mathbb{R}^p\) es medible p.c.t \(y \in \mathbb{R}^q\).
			\item Las funciones \(x \mapsto \lambda_q (E_x)\) e \(y \mapsto \lambda_p(E^y)\), definidas c.p.d en \(\mathbb{R}^p\) y \(\mathbb{R}^q\) respectivamente, son medibles y se verifica que
			\[
				\lambda_N (E) = \int_{\mathbb{R}^p} \lambda_q (x) dx = \int_{\mathbb{R}^q} \lambda_p(E^y) dy
			\]
		\end{enumerate}
		\end{teorema*}
		
		\bigskip
		
		\item Teorema de Fubini
		
		El teorema de Tonelli se utiliza principalmente como criterio de integrabilidad. Pero lo que ahora realmente nos interesa es comprobar que, supuesto que \(f\) sea integrable en \(\mathbb{R}^N\), su integral pueda calcularse también usando sus integrales iteradas. Eso es precisamente lo que afirma el siguiente teorema:
		\begin{teorema*}[Fubini]
		Para toda función \(f \in \mathcal{L}_1 (\mathbb{R}^N\) se tiene:
		\begin{enumerate}[label=(\textit{\roman*})]
			\item \(f_x \in \mathcal{L}_1 (\mathbb{R}^q)\) p.c.t \(x \in \mathbb{R}^p\) y \(f^y \in \mathcal{L}_1 (\mathbb{R}^p)\) p.c.t \(y \in \mathbb{R}^q\)
			\item Las funciones \(\varphi\) y \(\Psi\), definidas c.p.d en \(\mathbb{R}^p\) y \(\mathbb{R}^q\) respectivamente, por
			\[
				\varphi (x) = \int_{\mathbb{R}^q} f(x,y)dy \quad \text{p.c.t} \quad x \in \mathbb{R}^p \qquad \text{y} \qquad \Psi (y) = \int_{\mathbb{R}^p} f(x,y) dx \quad \text{p.c.t} y \in \mathbb{R}^q
			\]
			verifican que \(\varphi \in \mathcal{L}_1 (\mathbb{R}^p)\) y \(\Psi \in \mathcal{L}_1 (\mathbb{R}^q)\) con:
			\[
				\int_{\mathbb{R}^N} f(x,y) d(x,y) = \int_{\mathbb{R}^p} \varphi (x) dx = \int_{\mathbb{R}^q} \Psi (y) dy
			\]
		\end{enumerate}
		\end{teorema*}
	
	\end{enumerate}
\end{enumerate}

\end{document}
