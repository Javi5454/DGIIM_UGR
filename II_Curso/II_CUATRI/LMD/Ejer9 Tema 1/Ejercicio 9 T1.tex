\documentclass[a4paper, 12pt]{article}

\usepackage{amsmath} %Todos los paquetes de matematicas
\usepackage{amsthm}
\usepackage{amssymb}
\usepackage{amsfonts}
\usepackage{amssymb}
\usepackage[utf8]{inputenc}
\usepackage[spanish, es-lcroman]{babel}
\usepackage{wrapfig} %Figuras flotantes
\usepackage{parselines}
\usepackage{enumitem}
\usepackage{xcolor}
\usepackage{graphicx}
\usepackage{subcaption}
\usepackage{pgfplots}
\usepackage{hyperref}
\usepackage{textcomp}
\usepackage[left=2cm,right=2cm,top=2cm,bottom=2cm]{geometry}


\providecommand{\abs}[1]{\lvert#1\rvert} %Valor absoluto
\providecommand{\norm}[1]{\lVert#1\rVert} %Norma

\newtheorem*{lema*}{Lema}
\newtheorem{teorema}{Teorema}
\newtheorem*{teorema*}{Teorema}
\newtheorem{corolario}{Corolario}
\newtheorem{ej}{Ejercicio}
\renewcommand{\qedsymbol}{\(\blacksquare\)}

\providecommand{\abs}[1]{\lvert#1\rvert} %Valor absoluto
\providecommand{\norm}[1]{\lVert#1\rVert} %Norma


\title{\textbf{Ejercicio 9 Relación Tema 1} \\ \textit{Lógica y Métodos discretos}}
\author{Javier Gómez López}
\date{\today}

\begin{document}

\maketitle

\setcounter{ej}{8}

\begin{ej}
Es cierto que de un número \(n_0\) en adelante se tiene que \(100^n < n!\). Encuéntrelo y demuestre por inducción lo dicho a partir de ese número \(n_0\).
\end{ej}

En primer lugar, estudiemos la convergencia de la serie \(\sum_{n \geq 1} \frac{n!}{100^n}\). Para ello, sea \(\alpha_n = \frac{n!}{100^n}\) y apliquemos el criterio de d'Alembert o del cociente: 
\[
	\lim_{n \to \inf} \frac{\alpha_{n+1}}{\alpha_n} = \lim_{n \to \infty} \frac{\frac{(n+1)!}{100^{n+1}}}{\frac{n!}{100^n}} = \lim_{n \to \infty} \frac{n+1}{100} = \infty
\]

Vemos que dicho límite es mayor estricto que 1, por tanto el criterio del cociente nos dice que la serie \(\sum_{n \geq 1} \alpha_n\) diverge. \\

Que dicha serie es divergente, supone que:
\[
	\exists n_0 \in \mathbb{N} : n \geq n_0 \Rightarrow 100^n < n!
\]

Ahora, tratemos  de hallar dicho \(n_0\). Para ello, sabiendo que la función logaritmo es una función creciente, y por tanto, obtenemos la siguiente expresión:
\[
	\log (100^n) < \log (n!)
\]

Desarrollamos esta expresión:
\[
	n \cdot \log (100) < \log (n!) \underset{\underset{\text{Fórmula de Stirling}}{\downarrow}}{<} n \cdot \log (n) - n
\]
\[
	\log (100) < \log (n) -1
\]

Ahora aplicamos que la función exponencial es una función creciente:
\[
	e^{\log (100) + 1} < n
\]

Y deducimos que 
\[
	271 < n
\]

La fórmula de Stirling la hemos visto en la asignatura de Algorítmica, y se basa en la siguiente aproximación:
\[
	n! \simeq \sqrt{2 \pi n} \cdot \left( \frac{n}{e} \right)^n
\]

Cabe destacar que \(n_0 = 272\) no tiene por qué ser el mínimo natural que cumpla nuestro enunciado, debido a que la fórmula de Stirling es una aproximación, no una igualdad. Aún así, hemos encontrado un \(n_0 \in \mathbb{N}\) que cumple lo que queríamos probar. \\

Ahora pasamos a usar lo demostrado como caso base de nuestra hipótesis de inducción. Apliquemos el primer principio de inducción:
\begin{itemize}
	\item \textit{Supongamos \(n \in \mathbb{N}\), \(n \geq n_0\) tal que \(100^n < n!\) y probemos que \(100^{n+1} < (n+1)!\)}
\end{itemize}

Entonces, tenemos que 
\[
	100^{n+1} < (n+1)! \Rightarrow 100^n \cdot 100 < (n+1) \cdot n!
\]

Por la hipótesis de inducción, se verifica que \(100^n < n!\). Por otro lado, de \(100 < n +1\) sabemos que se cumple pues \(n \geq n_0\), pero \(n_0 = 272\). Por tanto, la desigualdad queda probada y hemos terminado.

\end{document}
