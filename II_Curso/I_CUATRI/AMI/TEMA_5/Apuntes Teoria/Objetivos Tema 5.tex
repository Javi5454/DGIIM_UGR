\documentclass[a4paper, 12pt]{article}

\usepackage{amsmath} %Todos los paquetes de matematicas
\usepackage{amsthm}
\usepackage{amssymb}
\usepackage{amsfonts}
\usepackage{amssymb}
\usepackage[utf8]{inputenc}
\usepackage[spanish, es-lcroman]{babel}
\usepackage{wrapfig} %Figuras flotantes
\usepackage{parselines}
\usepackage{enumitem}
\usepackage{xcolor}
\usepackage{graphicx}
\usepackage{subcaption}
\usepackage{pgfplots}
\usepackage{hyperref}
\usepackage{textcomp}
\usepackage[left=2cm,right=2cm,top=2cm,bottom=2cm]{geometry}


\providecommand{\abs}[1]{\lvert#1\rvert} %Valor absoluto
\providecommand{\norm}[1]{\lVert#1\rVert} %Norma

\newtheorem{teorema}{Teorema}
\renewcommand{\qedsymbol}{\(\blacksquare\)}


\title{\textbf{Objetivos de aprendizaje Tema 5} \\ \textit{Análisis Matemático I}}
\author{Javier Gómez López}
\date{\today}

\begin{document}

\maketitle

\begin{enumerate}[label=\textbf{\arabic*}.]

\item Conocer y comprender las siguientes definiciones:
	\begin{enumerate}[label=\textit{\alph*})]
	
	\item Sucesión de Cauchy y espacio métrico completo
	
	Si \(E\) es un espacio métrico con distancia \(d\), y \(x_n \in E\) para todo \(n \in \mathbb{N}\), se dice que \(\{x_n\}\) es una \textbf{sucesión de Cauchy} en \(E\), cuando:
	\[
		\forall \varepsilon > 0 \quad \exists m \in \mathbb{N} : p,q \geq m \Rightarrow d(x_p,x_q) < \varepsilon
	\]
	
	Es obvio que, en \(\mathbb{R}\) con la distancia usual, las sucesiones de Cauchy son las que ya conocíamos, que coinciden con las convergentes. En general, tenemos siempre una implicación:
	\begin{itemize}
		\item \textit{En cualquier espacio métrico, toda sucesión convergente es una sucesión de Cauchy}.
	\end{itemize}
	
	También es importante que, en general, el recíproco del resultado anterior no es cierto. \\
	
	Por otro lado, dado un espacio métrico \(E\) con distancia \(d\), se dice que \(d\) es una \textbf{distancia completa}, o también que \(E\) es un \textbf{espacio métrico completo}, cuando toda sucesión de Cauchy de puntos de \(E\) es convergente.
	
	El teorema de complitud de \(\mathbb{R}\), como su nombre indica, afirma que la distancia usual \(\mathbb{R}\) es completa, o que \(\mathbb{R}\) con la distancia usual es un espacio métrico completo.
	
	Diremos ahora que una norma \(||\cdot||\) en un espacio vectorial \(X\) es una \textbf{norma completa}, cuando es completa con la distancia \(d\) asociada, definida por \(d(x,y) = ||x-y||\) para cada \(x,y \in X\). Un \textbf{espacio de Banach} es un espacio normado que, con la distancia asociada a su norma, es un espacio métrico completo. Por otro lado, dado un espacio pre-hilbertiano cuya norma, la asociada a su producto escalar, es completa, recibe el nombre de \textbf{espacio de Hilbert}.
	
		
	\medskip
	
	\item Función uniformemente continua
	
	En lo que sigue fijamos dos espacios métricos \(E\) y \(F\), cuyas distancias se denotan ambas por \(d\), y una función \(f: E \rightarrow F\). La continuidad de \(f\) se expresa en la forma
	\[
		\forall x \in E \text{ } \forall \varepsilon > 0 \text{ } \exists \delta > 0 : y \in E, \text{ }d(x,y) < \delta \Rightarrow d(f(x),f(y)) < \varepsilon
	\]
	donde sabemos que \(\delta\) puede depender tanto de \(\varepsilon\) como del punto \(x \in E\) considerado. Pues bien, tendremos continuidad uniforme cuando podamos conseguir que \(\delta\) solo dependa de \(\varepsilon\).
	
	Por tanto, decimos que \(f\) es \textbf{uniformemente continua} cuando:
	\[
		\forall \varepsilon > 0 \text{ } \exists \delta > 0 : x,y \in E, \text{ } d(x,y) < \delta \Rightarrow d(f(x),f(y)) < \varepsilon
	\]
	
	Caracterizamos fácilmente la continuidad uniforme en términos de sucesiones:
	\begin{itemize}
		\item \textit{Si f es uniformemente continua y \(\{x_n\}\), \(\{y_n\}\) son sucesiones de puntos de E verificando que \(\{d(x_n, y_n)\} \rightarrow 0\), entonces \(\{d(fx_n),f(y_n))\} \rightarrow 0\). El recíproco también es cierto, más aún: si f no es uniformemente continua, existen dos sucesiones \(\{x_n\}\) e \(\{y_n\}\) de puntos de E y existe un \(\varepsilon > 0\) tales que \(d(x_n,y_n) < 1/n\) para todo \(n \in \mathbb{N}\), pero también se tiene que \(d(f(x_n),f(y_n)) \geq \varepsilon\) para todo \(n \in \mathbb{N}\)}
	\end{itemize}
	
\medskip

	\item Función lipschitziana y constante de Lipschitz
	
	Si \(E\) y \(F\) son espacios métricos, una función \(f:E \rightarrow F\) es \textbf{lipschitziana} cuando existe una constante \(M \in \mathbb{R}_0^+\) tal que:
	\begin{equation}\label{lipschitz}
	d(f(x),f(y)) \leq M d(x,y) \qquad \forall x,y \in E
	\end{equation}
	
	Es evidente que toda función lipschitziana es uniformemente continua, y sabemos que el recíproco es falso.
	
	La mínima constante \(M_0\) que verifica (\ref{lipschitz}) es la \textbf{constante de Lipschitz} de \(f\), que viene dada por
	\[
		M_0 = \text{sup} \left\{ \frac{d(f(x),f(y))}{d(x,y)} : x,y \in E, x \neq y \right\}
	\]
	
	Cuando \(M_0 \leq 1\) se dice que \(f\) es \textbf{no expansiva}. Cuando se tiene de hecho \(M_0 < 1\) decimos que \(f\) es \textbf{contractiva}.
	\end{enumerate}
	
\bigskip

	\item Conocer y comprender los siguientes resultados:
	
	\begin{enumerate}[label=\textit{\alph*})]
		\item Complitud de \(\mathbb{R}^N\)
		
		En el ambiente particular de los espacio normados tenemos una ventaja que no teníamos en espacios métricos cualesquiera:
	\begin{itemize}
		\item \textit{Dos normas equivalentes en un mismo espacio vectiral dan lugar a las mismas sucesiones de Cauchy. Por tanto, toda norma equivalente a una norma completa es completa.}
	\end{itemize}
	
		\begin{teorema}
		Todo espacio normado de dimensión finita es un espacio de Banach. Por tanto, el espacio euclídeo N-dimensional es un espacio de Hilbert.
		\end{teorema}
		
		\begin{proof}
		Basta, por ejemplo, probar que la norma del máximo en \(\mathbb{R}^N\) es completa. Si \(\{x_n\}\) es una sucesión de Cauchy de vectores de \(\mathbb{R}^N\), le aplicamos la desigualdad que ya hemos usado varias veces:
		\[
			| x_p(k) - x_q (k)| \leq ||x_p - x_q ||_{\infty} \quad \forall p,q \in \mathbb{N}, \forall k \in \Delta_N
		\]
		
		Deducimos que, para cada \(k \in \Delta_N\), la sucesión de números reales \(\{x_n(k)\}\) es de Cauchy luego, por el teorema de complitud de \(\mathbb{R}\), es convergente. Por tanto \(\{x_n\}\) es convergente.
		\end{proof}
		
		\begin{itemize}
			\item \textit{Sea E un espacio métrico y A un subespacio métrico de E. Se tiene:}
			
			\begin{enumerate}[label=(\textit{\roman*})]
				\item \textit{Si A es completo, entonces A es un subconjunto cerrado de E.}
				\item \textit{Si E es completo y A es un subconjunto cerrado de E, encontes A es completo}.
			\end{enumerate}
		\end{itemize}
		
	\medskip
	
	\item Versión general del teorema de Heine
	
	\begin{teorema}[\textbf{Heine}]
		Sean E y F dos espacios métricos y \(f: E \rightarrow F\) una función continua. Si E es compacto, entonces f es uniformemente continua.
	\end{teorema}
	
	\begin{proof}
	Por reducción al absurdo, suponemos que \(f\) no es uniformemente continua. Existen sucesiones \(\{x_n\}\) e \(\{y_n\}\) de puntos de \(E\), y un \(\varepsilon > 0\), tales que, para todo \(n \in \mathbb{N}\) se tiene
	\[
		d(x_n.y_n) < 1/n \qquad \text{y} \qquad d(f(x_n),f(y_n)) \geq \varepsilon
	\]
	
	Por ser \(E\) compacto, tenemos una sucesión parcial \(\{x_{\sigma (n)}\}\) que converge a un punto \(x \in E\). Puesto que \(\{d(x_{\sigma(n)}, y_{\sigma (n)})\} \rightarrow 0\), deducimos que también \(\{y_{\sigma (n)}\} \rightarrow x\). Como \(f\) es continua, tenemos que \(\{f(x_{\sigma (n)})\} \rightarrow f(x)\) y \(\{f(y_{\sigma (n)})\} \rightarrow f(x)\), luego \(\{d(f(x_{\sigma (n)}), f(y_{\sigma (n)}))\} \rightarrow 0\), lo cual es una contradicción, ya que \\ \(d(f(x_{\sigma (n)}), f(y_{\sigma (n)})) \geq \varepsilon\) para todo \(n \in \mathbb{N}\).
	\end{proof}
	\end{enumerate}

\bigskip

	\item Conocer los siguientes resultados, incluyendo su demostración:
	
		\begin{enumerate}[label=\textit{\alph*})]
			\item Teorema del punto fijo
			
			\begin{teorema}[\textbf{Punto fijo de Banach}]
			Sea E un espacio métrico completo y \(f: E \rightarrow E\) una aplicación contractiva. Entonces f tiene un único punto fijo, es decir, existe un único punto \(x \in E\) tal que \(f(x) =x\).			
			\end{teorema}
			
			\begin{proof}
			Fijamos \(x_0 \in E\) arbitrario y definimos por inducción una sucesión \(\{x_n\}\) de puntos de \(E\), tomando \(x_1 = f(x_0)\) y \(x_{n+1} = f(x_n)\) para todo \(n \in \mathbb{N}\). Probemos que esta sucesión es convergente y su límite será el punto fijo que buscamos.
			
			Si \(\alpha < 1\) es la constante de Lipshitz de \(f\) y \(\rho = d(x_0, x_1)\), comprobamos por inducción que
			\begin{equation}\label{hipotesis_induccion}
				d(x_n, x_{n+1}) \leq \alpha^n \rho \qquad \forall n \in \mathbb{N}
			\end{equation}
			
			En efecto, tenemos \(d(x_1,x_2) = d(f(x_0),f(x_1)) \leq \alpha d(x_0, x_1) = \alpha \cdot \rho\) y, suponiendo que (\ref{hipotesis_induccion}) se verifca para un \(n  \in \mathbb{N}\), deducimos que
			\[
				d(x_{n+1},x_{n+2}) = d(f(x_n),f(x_{n+1})) \leq \alpha d(x_n, x_{n+1}) \leq \alpha \cdot \alpha^n \cdot \rho = \alpha^{n+1} \cdot \rho
			\]
			
			Ahora, para cualesquiera \(n,k \in \mathbb{N}\) tenemos
			\begin{equation}\label{desigualdad}
				d(x_n, x_{n+k}) \leq \sum_{j=0}^{k-1} d(x_{n+j},x_{n+j+1}) \leq \rho \sum_{j=0}^{k-1} \alpha^{n+j} \leq \rho \cdot \alpha^n \cdot \sum_{j=0}^{\infty} \alpha^j = \frac{\rho \cdot \alpha^n}{1 - \alpha}
			\end{equation}
			
			De la desigualdad (\ref{desigualdad}) deduciremos fácilmente que \(\{x_n\}\) es una sucesión de Cauchy. En efecto, dado \(\varepsilon > 0\), como \(\{\alpha^n\} \rightarrow 0\), existe \(m \in \mathbb{N}\) tal que, para \(n \geq m\) se tiene que \(\rho \alpha^n < \varepsilon (1 - \alpha)\). Entonces, para \(p,q \geq m\), suponiendo sin perder generalidad que \(p < q\), usamos (\ref{desigualdad}) con \(n = p\) y \(k = q - p\) para obtener
			\[
				d(x_p,x_q) = d(x_n, x_{n+k}) \leq \frac{\rho \alpha^n}{1 - \alpha} < \varepsilon
			\]
			
			Como por hipótesis \(E\) es completo, tenemos \(\{x_n\} \rightarrow x \in E\), luego \(\{f(x_n)\} = \{x_{n+1}\} \rightarrow x\). Pero \(f\) es continua, luego \(\{f(x_n)\} \rightarrow f(x)\), y concluimos que \(f(x) = x\). Finalmente, si \(y \in E\) es otro punto fijo de \(f\), se tiene que \(d(x,y) = d(f(x),f(y)) \leq \alpha d(x,y)\), luego \((1 - \alpha) d(x,y) \leq 0\). Como \(\alpha < 1\), deducimos que \(d(x,y) \leq 0\), es decir \(x = y\).
			\end{proof}
			
			\medskip
			
			\item Caracterización de la continuidad de una aplicación lineal
			
			\begin{itemize}
				\item \textit{Sean X,Y dos espacios normados y sea \(T: X \rightarrow Y\) una aplicación lineal. Las siguientes afirmaciones son equivalentes:}
				
				\begin{enumerate}[label=(\textit{\roman*})]
					\item \textit{T es continua}
					\item \textit{Existe una constante \(M \in \mathbb{R}_0^+\) tal que \(||T(x)|| \leq M ||x||\) para todo \(x \in X\)}
				\end{enumerate}
			\end{itemize}
			
			\begin{proof}
			\textit{(i) \(\Rightarrow\) (ii)}. Teniendo en cuenta que \(T(0) = 0\), la continuidad de \(T\) en 0 nos dice que
			\[
				\exists \delta > 0 : z \in X, ||z|| \leq \delta \Rightarrow ||T(z)|| < 1
			\]
			
			Dado \(x \in X \setminus \{0\}\), tomando \(z = \frac{\delta x}{2 ||x||}\) tenemos claramente \(||z|| = \delta / 2 < \delta\), luego
			\[
				||T(x)|| = \frac{2 ||x||}{\delta} ||T(z)|| \leq \frac{2}{\delta} ||x||
			\]
			
			Como esta desigualdad es obvia cuando \(x=0\), hemos probado \textit{(ii)} con \(M = 2 / \delta\).
			
			\textit{(ii) \(\Rightarrow\) (i)}. Para cualesquiera \(u,v \in X\), de \textit{(ii)} deducimos claramente que
			\[
				||T(u) - T(v)|| = ||T(u-v)|| \leq M ||u - v||
			\]
			lo que prueba que \(T\) es lipschitziana, luego continua.
			\end{proof}
			
			\begin{itemize}
				\item \textit{Si X es un espacio normado de dimensión finita, toda aplicación lineal de X en cualquier otro espacio es normado, es continua.}
			\end{itemize}
			
			\begin{proof}
			Sea \(Y\) otro espacio normado y \(T: X \rightarrow Y\) una aplicación lineal. Definimos entonces una nueva norma \(||\cdot ||_T\) en \(X\), de la siguiente forma
			\[
				||x||_T = ||x|| + ||T(x)|| \qquad \forall x \in X
			\]
			
			Se comprueba rutinariamente que \(||\cdot||_T\) es efectivamente una norma en \(X\). Como \(X\) tiene dimensión finita, el teorema de Hausdorff nos dice que \(||\cdot||_T\) es equivalente a la norma de partida en \(X\), luego existe una constante \(\rho \in \mathbb{R}^+\) tal que \(||x||_T \leq \rho ||x||\) para todo \(x \in X\). Pero está bien claro que 
			\[
				||T(x)|| \leq ||x||_T \leq \rho ||x|| \qquad \forall x \in X
			\]
			y esto prueba que \(T\) es continua.
			\end{proof}
		\end{enumerate}
\end{enumerate}

\end{document}