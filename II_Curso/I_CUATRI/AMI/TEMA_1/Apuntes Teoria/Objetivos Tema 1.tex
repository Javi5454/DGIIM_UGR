\documentclass[a4paper, 12pt]{article}

\usepackage{amsmath} %Todos los paquetes de matematicas
\usepackage{amsthm}
\usepackage{amssymb}
\usepackage{amsfonts}
\usepackage{amssymb}
\usepackage[utf8]{inputenc}
\usepackage[spanish]{babel}
\usepackage{wrapfig} %Figuras flotantes
\usepackage{parselines}
\usepackage{enumitem}
\usepackage{xcolor}
\usepackage{graphicx}
\usepackage{subcaption}
\usepackage{pgfplots}
\usepackage{hyperref}
\usepackage{textcomp}
\usepackage[left=2cm,right=2cm,top=2cm,bottom=2cm]{geometry}


\providecommand{\abs}[1]{\lvert#1\rvert} %Valor absoluto
\providecommand{\norm}[1]{\lVert#1\rVert} %Norma

\title{\textbf{Objetivos de aprendizaje Tema 1} \\ \textit{Análisis Matemático I}}
\author{Javier Gómez López}
\date{\today}

\begin{document}

\maketitle

\begin{enumerate}[label=\textbf{\arabic*}.]

\item Conocer y comprender las siguientes definiciones:
	\begin{enumerate}[label=\textit{\alph*})]
	
	\item Espacio euclídeo \textit{N}-dimensional
	
\(\mathbb{R}^N\) es el producto cartesiano de \textit{N} copias de \(\mathbb{R}\), es decir, el conjunto de todas las posibles \textit{N}-uplas de números reales:
\[
	\mathbb{R}^N = \mathbb{R} \times \mathbb{R} \times \overset{(N)}{....} \times \mathbb{R} = \left\{ (x_1, x_2, \dotsc , x_N) : x_1, x_2, \dotsc, x_N \in \mathbb{R} \right\}
\]

En \(\mathbb{R}^N\) disponemos de las operaciones de \textbf{suma} y \textbf{producto por escalares}, que vienen definidas, para \(x = (x_1, x_2, \dotsc, x_N) \in \mathbb{R}^N\), \(y = (y_1, y_2, \dotsc, y_N) \in \mathbb{R}^N\) y \(\lambda \in \mathbb{R}\), por
\[
	x+y = (x_1 + y_1, x_2 + y_2, \dotsc, x_N + y_N)
\]
\[
	\lambda x = (\lambda x_1, \lambda x_2, \dotsc, \lambda x_N)
\]

Sabemos que, con estas operaciones, \(\mathbb{R}^N\) tiene estructura de \textit{espacio vectorial}. El espacio vectorial \(\mathbb{R}^N\) tiene \textit{dimensión N}. Destacamos la \textit{base usual}, \(\Phi = \{e_1, e_2, \dotsc, e_N\}\) donde para cada \(k \in \Delta_N\), llamamos \(e_k\) a la \textit{N}-upla cuya \(k\)-ésima componente es 1 y las demás se anulan, es decir:
\[
	e_k(k) = 1 \qquad \text{y} \qquad e_k(j) = 0 \quad \forall j \in \triangle _N \setminus \{k\}
\]

Recordemos que \(\mathbb{R}^N\) es, salvo isomorfismos, el \textit{único} espacio vectorial de dimensión \textit{N}. \\

Presentado \(\mathbb{R}^N\) como espacio vectorial, definamos ahora el concepto de \textbf{producto escalar}. El producto escalar de dos vectores \(x = (x_1, \dotsc, x_N) \in \mathbb{R}^N\) e \(y = (y_1, \dotsc, y_n) \in \mathbb{R}^N\) es el número real \((x | y)\) dado por:
\[
	(x|y) = \sum_{k=1}^{N} x_k y_k
\]
que verifica las siguientes propiedades
\[
\begin{array}{cc}
\text{P1)} & (\lambda u + \mu v |y)  =  \lambda (u|y) + \mu (v|y) \\
\text{P2)} & (x|y)  =  (y|x) \\
\text{P3)} & (x|x) > 0 \quad \forall x \in \mathbb{R}^N \setminus \{0\}
\end{array}
\]

Si asociamos \(\mathbb{R}^N\) con el producto escalar definido es un espacio pre-hilbertiano, conocido como el espacio euclídeo N-dimensional. \\
\medskip 

	\item Espacio pre-hilbertiano 
	
Un \textbf{espacio pre-hilbertiano} es, por definición, un espacio vectorial dotado de un producto escalar. El \textit{producto escalar de dos vectores} \(x = (x_1, x_2, \dotsc, x_N) \in \mathbb{R}^N\) e \(y = (y_1,y_2, \dotsc , y_N) \in \mathbb{R}^N\) es, por definición el número real \(\left( x | y \right)\) dado por
\begin{equation}\label{producto_escalar}
	\left( x | y \right) = \sum_{k=1}^{N} x_k y_k = \sum_{k=1}^{N} x(k) y(k)
\end{equation}
y decimos también que la aplicación \(\left( \cdot | \cdot \right) : \mathbb{R}^N \times \mathbb{R}^N \rightarrow \mathbb{R}\), dada por \((x,y) \mapsto (x|y)\) para cualesquiera \(x,y \in \mathbb{R}^N\), es el \textbf{producto escalar} en \(\mathbb{R}^N\). 

En general, el producto escalar en \(\mathbb{R}^N\) tiene tres propiedades claras:
\[
\begin{array}{lll}
\text{(\textbf{P.1})} \quad &(\lambda u + \mu v | y) = \lambda (u | y) +  \mu (v | y) \quad & \forall u,v,y \in \mathbb{R}^N \text{ , } \forall \lambda , \mu \in \mathbb{R}^N \\
\text{(\textbf{P.2})} \quad &(x|y) = (y|x) \quad &\forall x,y \in \mathbb{R}^N \\
\text{(\textbf{P.3})} \quad & (x|x) > 0 \quad & \forall x \in \mathbb{R}^N \setminus \{0 \}
\end{array}
\]

Un \textbf{producto escalar} en un espacio vectorial \(X\) es una forma bilineal simétrica en \(X\), cuya forma cuadrática asociada es definida positiva. Por tanto, \(\mathbb{R}^N\) con el producto escalar definido como en (\ref{producto_escalar}) es un espacio pre-hilbertiano, que se conoce como el \textbf{espacio euclídeo} \(N\)-\textit{dimensional}.

\medskip

	\item Espacio normado
	
Se define la \textbf{norma de un vector} \(x \in X\) como la raíz cuadrada del producto escalar de \(x\) por sí mismo, es decir, el número real no negativo \(||x||\) dado por
\begin{equation}\label{norma_escalar}
	||x|| = (x|x)^{1/2}
\end{equation}

Una \textbf{norma} es un espacio vectorial \(X\) es una aplicación \(|| \cdot || : X \rightarrow \mathbb{R}\), que a cada \(x \in X\) hace corresponder un número real \(||x||\), verificando las tres condiciones siguientes:
\[
\begin{array}{lll}
\text{(\textbf{N.1})} \quad & \textit{Desigualdad triangular: }||x+y|| \leq ||x|| + ||y||  & \forall x,y \in X \\
\text{(\textbf{N.2})} \quad & \textit{Homogeneidad por homotecias: }||\lambda x|| = |\lambda| ||x|| & \forall x \in X, \quad \forall \lambda \in \mathbb{R} \\
\text{(\textbf{N.3})} \quad & \textit{No degeneración: } x \in X, \quad ||x|| = 0 \Longrightarrow x = 0 
\end{array}
\]

Un \textbf{espacio normado} es un espacio vectorial \(X\), en el que hemos fijado una norma \(|| \cdot ||\). En un mismo espacio vectorial \(X\) podemos tener varias normas distintas. Con cada una de esas normas, tendremos un espacio normado diferente.

Si \(|| \cdot ||\) es una norma en un espacio vectorial \(X\), usando la homogeneidad por homotecias con \(\lambda = -1\), vemos que \(||-x|| = ||x||\) para todo \(x \in X\). Pero entonces, la desigualdad triangular implica que
\[
	0 = || x + (-x)|| \leq ||x|| + ||-x|| = 2||x|| \quad \forall x \in X
\]
así que una norma nunca puede tomar valores negativos.

\medskip

	\item Espacio métrico

Dados dos puntos \(x,y \in \mathbb{R}^N\), el segmento que va de \(x\) a \(y\) se obtiene trasladando el que va del origen a \(y-x\), luego la longitud del vector \(y-x\) nos da la distancia de \(x\) a \(y\). 

Si \(X\) es un espacio normado, se define la \textit{distancia entre dos puntos } de \(X\) por
\begin{equation}\label{distancia}
	d(x,y) = || y-x || \quad \forall x,y \in X
\end{equation}
Obtenemos así una función de dos variables \(d: X \times X \rightarrow \mathbb{R}\), a partir de la cual podemos recuperar la norma de \(X\), puesto que evidentemente
\begin{equation}
	||x|| = d(0,x) \quad \forall x \in X
\end{equation}

Por tanto, una \textbf{distancia} en un conjunto no vacío \(E\) es una función \(d: E \times E \rightarrow \mathbb{R}\) verificando las siguientes condiciones:
\[
\begin{array}{lll}
\text{(\textbf{D.1})} \quad & \textit{Desigualdad triangular: }d(x,z) \leq d(x,y) + d(y,z)  & \forall x,y,z \in E \\
\text{(\textbf{D.2})} \quad & \textit{Simetría: } d(x,y) = d(y,x) & \forall x,y \in E \\
\text{(\textbf{D.2})} \quad & \textit{No degeneración: Para } x,y \in E, \text{ se tiene: } d(x,y) = 0 \Longleftrightarrow x = y
\end{array}
\]
	\end{enumerate}
	
\newpage

\item Conocer la relación entre los conceptos anteriores y los ejemplos que permitan distinguir entre ellos

Todo espacio pre-hilbertiano es un espacio normado, con la norma asociada a su producto escalar. En particular, el espacio euclídeo \(N\)-dimensional (que es un espacio pre-hilbertiano con producto escalar definido como en (\ref{producto_escalar})) es un espacio normado, \(\mathbb{R}^N\) con la norma euclídea.

Antes de ver otros ejemplos, conviene comentar que, en todo espacio normado \(X\), la norma de cada vector se interpreta siempre como la longitud que asignamos al vector. 

La norma euclídea en \(\mathbb{R}\) coincide con el valor absoluto. Salvo cambios de escala, el valor absoluto es la única norma que cabe considerar en \(\mathbb{R}\). 

Para \(N>1\), es sencillo definir en \(\mathbb{R}^N\) normas que no son proporcionales a la euclídea. Concretamente, podemos definir la \textbf{norma del máximo}, \(|| \cdot ||_{\infty}\), y la \textbf{norma de la suma}, \(|| \cdot ||_1\), escribiendo, para \(x=(x_1,x_2,\dotsc,x_N) \in \mathbb{R}^N\),
\begin{equation}
	||x||_{\infty} = \text{máx} \left\{ |x_k| : k \in \triangle _N \right \} \quad \text{ y } \quad ||x||_1 = \sum_{k=1}^{N} |x_k|
\end{equation}
Estas dos normas no proceden de ningún producto escalar en \(\mathbb{R}^N\).

Consideremos de nuevo el espacio vectorial \(C[0,1]\), formado por las funciones continuas del intervalo [0,1] en \(\mathbb{R}\). Conocemos en este espacio un producto escalar que lo convierte en un espacio pre-hilbertiano, luego es un espacio normado con la norma dada por
\[
	||x|| = (x|x)^{1/2} = \left( \int_{0}^{1} x(t)^2 dt \right)^{1/2} \quad \forall x \in C[0,1]
\]
Por otro lado, para \(x \in C[0,1]\), tenemos que
\begin{equation}
	||x||_{\infty} = \text{máx} \left\{ |x(t)| : t \in [0,1] \right\} \quad \text{ y } \quad ||x||_1 = \int_{0}^{1} |x(t)| dt
\end{equation}

Estas normas tampoco proceden del producto escalar, y por tanto, tenemos dos ejemplos de espacios normados de dimensión infinito, que no son espacios pre-hilbertianos.

\medskip

\item Conocer los siguientes resultados, incluyendo su demostración:
	\begin{enumerate}[label=\textit{\alph*})]
		\item Desigualdad de Cauchy-Schwartz
		
		\textit{En todo espacio pre-hilbertiano \(X\), se tiene:}
\begin{equation}\label{Cauchy-Schwartz}
	|(x|y)| \leq ||x|| ||y|| \quad \forall x,y \in X
\end{equation}
\textit{Además, se verifica la igualdad si, y sólo si, x e y son linealmente dependientes} \\

\textbf{Demostración}. Si \(x,y \in X\) son linealmente dependientes, se tendrá que \(y=0\), o bien, \(x = \lambda y\) con \(\lambda \in \mathbb{R}\). En el primer caso la igualdad buscada es trivial y en el segundo basta usar (\textbf{N.2}):
\[
	|(x|y)| = |(\lambda y | y)| = |\lambda| ||y||^2 = ||x|| ||y||
\]

Supongamos pues que \(x,y \in X\) son linealmente independientes, y en particular no nulos, para probar la desigualdad estricta en (\ref{Cauchy-Schwartz}). Para todo \(\lambda \in \mathbb{R}\), tenemos
\[
	0 < (x - \lambda y | x - \lambda y) = ||x||^2 - 2 \lambda (x|y) + \lambda^2 ||y||^2
\]
Podemos suponer que \((x|y) \neq 0\), pues en otro caso la desigualdad buscada es obvia, lo que nos permite tomar \(\lambda = \frac{||x||^2}{(x|y)}\), obteniendo
\[
	0 < -||x||^2 + \frac{||x||^4 ||y||^2}{(x|y)^2} \Rightarrow ||x||^2 (x|y)^2 < ||x||^4 ||y||^2
\]
Basta ahora dividir ambos miembros por \(||x||^2 > 0\) y tomar raíces cuadradas.

\medskip

	\item Desigualdad triangular para la norma de un espacio pre-hilbertiano
	
Para cada \(x,y \in X\), basta pensar que 
\[
	||x+y||^2 = ||x||^2 + 2(x|y) + ||y||^2 \leq ||x||^2 +2 ||x|| ||y|| + ||y||^2 = (||x|| + ||y||)^2
\]
donde hemos usado la desigualdad de Cauchy-Schwartz. Podemos tomar raíces cuadradas en ambos miembros de la desigualdad, puesto que son números reales positivos, y por tanto quedaría probada la desigualdad triangular.
	\end{enumerate}

\end{enumerate}

\end{document}