\documentclass[a4paper, 12pt]{article}

\usepackage{amsmath} %Todos los paquetes de matematicas
\usepackage{amsthm}
\usepackage{amsfonts}
\usepackage{amssymb}
\usepackage[utf8]{inputenc}
\usepackage[spanish]{babel}
\usepackage{wrapfig} %Figuras flotantes
\usepackage{parselines}
\usepackage{enumitem}
\usepackage{xcolor}
\usepackage{graphicx}
\usepackage{subcaption}
\usepackage{tikz}
\usepackage{pgfplots}
\usepackage{hyperref}
\usepackage{textcomp}
\usepackage[left=2cm,right=2cm,top=2cm,bottom=2cm]{geometry}

\theoremstyle{definition}
\newtheorem{ej}{Ejercicio}

\providecommand{\abs}[1]{\lvert#1\rvert} %Valor absoluto
\providecommand{\norm}[1]{\lVert#1\rVert} %Norma

\title{Ejercicio 16/9/2021}
\author{Javier Gómez López}
\date{\today}

\begin{document}

\maketitle

\begin{ej}
Sea \((X,d)\) un espacio métrico (\(d\) distancia en \(X\)). Tomemos \(r > 0\). Definimos \(d_r : X \times X \longrightarrow \mathbb{R}\) por
\[
	d_r(x,y) = r \cdot d(x,y), \quad x,y \in X
\]

Probar que \(d_r\) es una distancia en \(X\). \\

Debemos probar las 3 propiedades que definen una distancia:
\begin{enumerate}
	\item \(d(x,y) = 0 \Longleftrightarrow x=y\)
	
\framebox[0.7cm][c]{\(\Rightarrow\)} 
\[
	d_r(x,y) = 0 \Rightarrow r \cdot d(x,y)=0 \underset{\underset{r>0}{\uparrow}}{\Rightarrow} d(x,y) = 0
\]
Puesto que por definición \(d\) es una distancia, \(x=y\).

\framebox[0.7cm][c]{\(\Leftarrow\)}
\[
	x=y \Rightarrow d(x,y) = 0 \Rightarrow d_r(x,y) = \underset{\underset{0}{\uparrow}}{d(x,y)} \cdot r \Rightarrow d_r(x,y) = 0
\]

	\item \(d(x,y) = d(y,x)\)
\[
	d_r(x,y) = r \cdot d(x,y) = r \cdot d(y,x) = d_r(y,x)
\]

	\item \(d(x,y) \leq d(x,z) + d(y,z)\) \(\forall x,y,z \in X\)

Primero de todo definamos
\[
	d_r(x,y) = r \cdot d(x,y) \quad d_r(x,z) = r \cdot d(x,z) \quad d_r(y,z) = r \cdot d(y,z)
\]
\[
	d_r(x,y) \leq d_r(x,z) + d_r(y,z) \Rightarrow r \cdot d(x,y) \leq r \cdot ((d(x,z) + d(y,z)) \Rightarrow d(x,y) \leq d(x,z) + d(y,z)
\]
\end{enumerate}

Y queda probado que \(d_r\) es una distancia en \(X\).
\end{ej}
\end{document}