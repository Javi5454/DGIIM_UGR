\documentclass[a4paper, 12pt]{article}

\usepackage{amsmath} %Todos los paquetes de matematicas
\usepackage{amsthm}
\usepackage{amsfonts}
\usepackage{amssymb}
\usepackage[utf8]{inputenc}
\usepackage[spanish]{babel}
\usepackage{wrapfig} %Figuras flotantes
\usepackage{parselines}
\usepackage{enumitem}
\usepackage{xcolor}
\usepackage{graphicx}
\usepackage{subcaption}
\usepackage{tikz}
\usepackage{pgfplots}
\usepackage{hyperref}
\usepackage{textcomp}
\usepackage[left=2cm,right=2cm,top=2cm,bottom=2cm]{geometry}

\theoremstyle{definition}
\newtheorem{ej}{Ejercicio}

\providecommand{\abs}[1]{\lvert#1\rvert} %Valor absoluto
\providecommand{\norm}[1]{\lVert#1\rVert} %Norma

\title{Ejercicio 16/9/2021}
\author{Javier Gómez López}
\date{\today}

\begin{document}

\maketitle

\begin{ej}
Sean \(d_2\) y \(d = \) distancia discreta. Prueba que no son métricamente equivalentes en \(\mathbb{R}^n\).

Dos distancias son equivalentes si 
\begin{equation} \label{met_equivalentes}
\exists \alpha , \beta > 0 : \alpha \cdot d(x,y) \leq d'(x,y) \leq \beta \cdot d(x,y) \quad \forall x,y \in X
\end{equation}

El caso en el que \(x=y\) cumple la desigualdad y es una comprobación trivial. Recordemos que \(d_2\) es una distancia en \(\mathbb{R}^n\) que procede de una norma y por tanto es no acotada. Por otro lado, tomando \(x \neq y\), tenemos que de ser métricamente equivalentes, tendríamos que 
\[
	d_2 (x,y) \leq \beta \cdot d(x,y) = \beta \Rightarrow d_2(x,y) \leq \beta
\]
lo cual es un absurdo y quedaría demostrado.
\end{ej}

\bigskip

\begin{ej}
Sea \((X,d)\) un espacio métrico.
\begin{enumerate}
	\item Probar que \(d' = \text{min}\{1,d\}\) es una distancia en \(X\).
	
Para ello hay que probar las tres propiedades que deben de cumplir todas las distancias.
	\begin{enumerate}
		\item \(d'(x,y) = 0 \Longleftrightarrow \text{min}\{1,d(x,y)\} = 0 \Longleftrightarrow d(x,y) = 0 \underset{\underset{d \text{ es una distancia}}{\uparrow}}{\Longleftrightarrow} x=y\)
		\item \(d'(x,y) = \text{min}\{1,d(x,y)\} = \text{min}\{1,d(y,x)\} = d'(y,x)\) \(\forall x,y \in X\).
		\item Queremos comprobar la desigualad triangular:
		\[
			d'(x,y) \leq d'(x,z) + d'(z,y)
		\] 
		Podemos afirmar que \(d' \leq 1\). Si \(d'(x,z)=1\) o \(d'(z,y)=1\) entonces
		\[
			d'(x,z) + d'(z,y) \geq 1 \geq d'(x,y)
		\]
		
		Ahora supongamos \(d'(x,z) <1 \) y \(d'(z,y) < 1\). Por tanto
		\[
		d'(x,z) = d(x,z) \text{ y } d'(z,y) = d(z,y)
		\]
		\[
			d'(x,y) \leq d(x,y) \leq d(x,z) + d(z,y) = d'(x,z) + d'(z,y)
		\]
		
		Y queda demostrado que \(d'(x,y) \leq d'(x,z) + d'(z,y)\) \(\forall x,y,z \in X\).
	\end{enumerate}
	
	\item Si \((X,d)\) es no acotado, \(d, d'\) no son métricamente equivalentes.
	
	Recordemos (\ref{met_equivalentes}) para definir cuando dos distancias son métricamente equivalentes. Para la parte de la derecha de la desigualdad podemos tomar \(\beta = 1\). Si existiera \(\alpha > 0 : \alpha \cdot d \leq d' \Rightarrow \alpha \cdot d(x,y) \leq d'(x,y) \leq 1\) \(\forall x,y \in X\), obtendríamos que
	\[
		d(x,y) \leq \frac{1}{\alpha} \quad \forall x,y \in X
	\]
	lo cual es un absurdo. Por tanto, \(\nexists \alpha : \alpha \cdot d \leq d'\) y queda probado que no son métricamente equivalentes.
\end{enumerate}
\end{ej}
\end{document}